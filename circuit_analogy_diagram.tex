%     This Tex written by Eric J. Mott is licensed under The MIT License (MIT)
%
%     Copyright (c) 2014 Eric J. Mott
%
%     Permission is hereby granted, free of charge, to any person obtaining a copy of this software and associated documentation files (the "Software"), to deal in the Software without restriction, including without limitation the rights to use, copy, modify, merge, publish, distribute, sublicense, and/or sell copies of the Software, and to permit persons to whom the Software is furnished to do so, subject to the following conditions:
%
%     The above copyright notice and this permission notice shall be included in all copies or substantial portions of the Software.
%
%     THE SOFTWARE IS PROVIDED "AS IS", WITHOUT WARRANTY OF ANY KIND, EXPRESS OR IMPLIED, INCLUDING BUT NOT LIMITED TO THE WARRANTIES OF MERCHANTABILITY, FITNESS FOR A PARTICULAR PURPOSE AND NONINFRINGEMENT. IN NO EVENT SHALL THE AUTHORS OR COPYRIGHT HOLDERS BE LIABLE FOR ANY CLAIM, DAMAGES OR OTHER LIABILITY, WHETHER IN AN ACTION OF CONTRACT, TORT OR OTHERWISE, ARISING FROM, OUT OF OR IN CONNECTION WITH THE SOFTWARE OR THE USE OR OTHER DEALINGS IN THE SOFTWARE.

\documentclass[12pt, oneside]{article}   	% use "amsart" instead of "article" for AMSLaTeX format
\usepackage{circuitikz}		
\usepackage{pythontex}

\begin{document}

% =========================================== new section =========================================== 
\section{Velocity Diagram}

\hspace*{-1.3in} % Shifts figure left outside of the default article class margins.
\begin{circuitikz}[font=\tiny] % Make default font size for text in the resistive network diagram tiny.
\ctikzset { label/align = straight }

\def\hspc{3.2} % horizontal spacing
\def\vspc{4} % vertical spacing

\begin{pycode}
from pipe.pipe import pipe # Import the module that runs the python program pipe.py, which analyzes the fluidics network.
ans = pipe() # Run the function that returns all velocities and mass flow rates in a python dictionary for the legs of the fluidics network.

nodesMainStart = ["","","","A"] # Array to label the initial node of the supply leg of the resistive network diagram.
nodesMainEnd = ["E","D","C","B"] # Array to label the nodes of the primary manifold of the resistive network diagram.
nodesHorizontal = [["U","V","W","X","Y"],["P","Q","R","S","T"],["K","L","M","N","O"],["F","G","H","I","J"]] # Array to label the nodes of the rest of the resistive network diagram.
velMain = ["v_7","v_5","v_3","v_1"] # Array to label the primary legs of the resistive network diagram.
velHorizontal = [["v_8","v_40","v_42","v_44","v_46"],["v_6","v_30","v_32","v_34","v_36"],["v_4","v_20","v_22","v_24","v_26"],["v_2","v_10","v_12","v_14","v_16"]] # Array to label the horizontal legs of the resistive network diagram.
velVertical = [["v_41","v_43","v_45","v_47","v_49"],["v_31","v_33","v_35","v_37","v_39"],["v_21","v_23","v_25","v_27","v_29"],["v_11","v_13","v_15","v_17","v_19"]] # Array to label the vertical legs of the resistive network diagram.

# The following arrays are used in the second diagram, but established here for efficiency.
massMain = ["m_7","m_5","m_3","m_1"]  # Array to label the primary legs of the resistive network diagram.
massHorizontal = [["m_8","m_40","m_42","m_44","m_46"],["m_6","m_30","m_32","m_34","m_36"],["m_4","m_20","m_22","m_24","m_26"],["m_2","m_10","m_12","m_14","m_16"]] # Array to label the horizontal legs of the resistive network diagram.
massVertical = [["m_41","m_43","m_45","m_47","m_49"],["m_31","m_33","m_35","m_37","m_39"],["m_21","m_23","m_25","m_27","m_29"],["m_11","m_13","m_15","m_17","m_19"]] # Array to label the vertical legs of the resistive network diagram.

# Loop that generates the resistive network diagram and labels it with the proper velocities in each leg (i.e. resistor).
for j in range(0,4):
    for i in range(0,5):
        # Velocity diagram
        print(r"\draw (%s*\hspc,%s*\vspc) to [R=$%s$,*-*,i^>=$$] (%s*\hspc+\hspc,%s*\vspc) node[font=\small, anchor=south]{%s};" % (i, j, ans[velHorizontal[j][i]], i, j, nodesHorizontal[j][i]))
        print(r"\draw (%s*\hspc+\hspc,%s*\vspc-2) node[font=\scriptsize, anchor=north] {Z} to [R=$%s$,i<^=$$,o-*] (%s*\hspc+\hspc,%s*\vspc);" % (i, j, ans[velVertical[j][i]], i, j))
        
    # Velocity diagram
    print(r"\draw (0,%s*\vspc) node[font=\small, anchor=east]{%s} to [R=$%s$, i<^=$$, *-*] (0,%s*\vspc+\vspc) node[font=\small, anchor=east]{%s};" % (j, nodesMainEnd[j], ans[velMain[j]], j, nodesMainStart[j]))

\end{pycode}

\end{circuitikz}

% =========================================== new section =========================================== 
\section{Mass Flow Rate Diagram}

\hspace*{-1.3in}	
\begin{circuitikz}[font=\tiny]	
\ctikzset { label/align = straight }

\def\hspc{3.2} % horizontal spacing
\def\vspc{4} % vertical spacing

\begin{pycode}

for j in range(0,4):
    for i in range(0,5):
	# Mass flow rate diagram
        print(r"\draw (%s*\hspc,%s*\vspc) to [R=$%s$,*-*,i^>=$$] (%s*\hspc+\hspc,%s*\vspc) node[font=\small, anchor=south]{%s};" % (i, j, ans[massHorizontal[j][i]], i, j, nodesHorizontal[j][i]))
        print(r"\draw (%s*\hspc+\hspc,%s*\vspc-2) node[font=\scriptsize, anchor=north] {Z} to [R=$%s$,i<^=$$,o-*] (%s*\hspc+\hspc,%s*\vspc);" % (i, j, ans[massVertical[j][i]], i, j))

    # Mass flow rate diagram
    print(r"\draw (0,%s*\vspc) node[font=\small, anchor=east]{%s} to [R=$%s$, i<^=$$, *-*] (0,%s*\vspc+\vspc) node[font=\small, anchor=east]{%s};" % (j, nodesMainEnd[j], ans[massMain[j]], j, nodesMainStart[j]))
    
\end{pycode}

\end{circuitikz}

\end{document}